\documentclass{report}
\usepackage[utf8]{inputenc}


\usepackage[a4paper, total={6in, 8in}]{geometry}


\title{Entwiklung Algorithmus Obere Halswirbelsäule}
\author{Lukas Hörnig}
\date{Oktober 2017}

\usepackage[square,sort,comma,numbers]{natbib}
\usepackage{graphicx}
\usepackage{hyperref}
\usepackage{gensymb}

\begin{document}

\maketitle


\section{Werte}
\subsection{Beschreibung}
\paragraph{ADI - Atlanto Dens Interval}

Das Atlanto-Dentale Intervall wird als der Abstand der Kortikalis der posterioren Kante des anterior Atlasbogens bis zur Kortikalis der ventralen Denskante auf der Verbindungslinie zwischen den Zentren des anterioren Atlasbogens und dem posterioren Atlasbogen in Millimetern gemessen. 
% Literatur
%\citep{Barker2006} \citep{Chang2009} \citep{Chen2011} \citep{Fielding1977} \citep{Gonzalez2004} \citep{Bono2007} \citep{Rojas2007} \citep{Monu1987} \citep{Liu2014}

\subsubsection{PADI}
Analog zum ADI wird mit dem Posterioren Atlanto-Dentalen Intervall der ABstand der Kortikalis der posterioren Kante des Dens bis zur Kortikals der ventralen Kante des posterioren Atlasbogens auf der Verbindungslinie zwischen den Zentren der Atlasbögen in Millimetern. 


% Literatur
% \citep{Bono2007} 

\paragraph{C0-C1 Translation} % (fold)
\label{par:c0_c1_translation}
Distanz in Millimetern zwischen dem Mittelpunkt des Basions zu einer vertikalen Verbindungsliene der Densachse.



% paragraph c0_c1_translation (end)
\paragraph{LMD - Lateral Mass Displacement }
Definiert als die Distanz in Millimetern zwischen 2 vertikalen Linien die von den lateralen Anteilen der Gelenkfortsätze des Atlas und des Axis gezeichnet werden.
% Literatur
%\citep{Lee1998} \citep{Bono2007}

\paragraph{LMI - Lateral Mass Index}

Der Lateral Mass Index wird als kürzeste Distanz zwischen den Mittelpunkten der Gelenkflächen C1-C2 im Coronaren oder sagitalen Schnitt in Millimetern gemessen. 
\subparagraph{Statistik}
Der Mittelwert beträgt 1.70 mm mit einer Range von 0.7–3.30 mm und einer Standardaabweichung von 0.48.
\subparagraph{Interpretation}
 Ab einem Wert von 2,6 mm soll bei zusammenhängendem Trauma eine Instabilität angenommen.

% Literatur
% \citep{Gonzalez2004} \citep{Chang2009}
 


\paragraph{Atlanto-Axial Rotation} % (fold)
\label{par:c1_c2_rotation}
Winkel in Grad zwischen der Line im transversal Schnitt vom Mittelpunkt des anterioren Axisbogens zum Mittelpunkt des Dornfortsatzes und einer Geraden durch die tranversalen Foramina des Atlas in transversaler Ebene.
Eine Rotation des Atlas auf dem Axis im Uhrzeigersinn wird als positiver Gradwert, eine Rotataion gegen den Uhrzeigersinn wird als negativer Wert definiert.

\subparagraph{Interpretation}
 Ein Wert  $>\pm 45 \degree$ definiert rotatorische Instabilität
% paragraph c1_c2_rotation (end)
\paragraph{Powers Ratio} % (fold)
\label{par:powers_ratio}
Quotient der Distanzen BDI/AO

\subparagraph{Statistik}
\begin{tabular}
dd
\end{tabular}
\subparagraph{Interpretation} % (fold)
\label{subp:}
> 0,9 Im CT als Zeichen Atlanto-Occipitaler Dislokation zu werten.

% subparagraph  (end)
% paragraph powers_ratio (end)

\paragraph{C1/C2 Angulation} % (fold)
\label{par:c1_c2_angulation}

% paragraph c1_c2_angulation (end)
\paragraph{}

\end{document}
